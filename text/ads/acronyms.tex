\addchap{Abkürzungsverzeichnis}

% WICHIG: Die längste Abkürzung (hier 'DHBW') muss als Parameter in Zeile 7 angegeben werden. Ansonsten
% kann es sein, dass die Darstellung des Abkürzungsverzeichnisses nicht richtig funktioniert.

\begin{acronym}[DHBW]
    \setlength{\itemsep}{0pt}
    \setlength{\parsep}{0pt}

    \acro{DHBW}{Duale Hochschule Baden-Württemberg}
    \acro{IT}{Informationstechnik}
    \acro{NAS}{Network Attached Storage}
    \acro{www}{World Wide Web}
\end{acronym}

% \ac{Abk.}   --> fügt die Abkürzung ein, beim ersten Aufruf wird zusätzlich automatisch die ausgeschriebene Version davor eingefügt
% \acs{Abk.}  --> fügt die Abkürzung ein
% \acf{Abk.}  --> fügt die Abkürzung UND die Erklärung ein
% \acl{Abk.}  --> fügt nur die Erklärung ein
% \acp{Abk.}  --> gibt Plural aus (angefügtes 's'); das zusätzliche 'p' funktioniert auch bei obigen Befehlen