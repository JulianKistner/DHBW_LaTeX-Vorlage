\addchap{Abkürzungsverzeichnis}

\begin{acronym}[UART]
    \setlength{\itemsep}{0pt}
    \setlength{\parsep}{0pt}

\acro{DHBW}{Duale Hochschule Baden-Württemberg}
\acro{IT}{Informationstechnik}
\acro{NAS}{Network Attached Storage}
\acro{www}{World Wide Web}

\end{acronym}

% \ac{Abk.}   --> fügt die Abkürzung ein, beim ersten Aufruf wird zusätzlich automatisch die ausgeschriebene Version davor eingefügt bzw. in einer Fußnote (hierfür muss in header.tex \usepackage[printonlyused,footnote]{acronym} stehen) dargestellt
% \acs{Abk.}  --> fügt die Abkürzung ein
% \acf{Abk.}  --> fügt die Abkürzung UND die Erklärung ein
% \acl{Abk.}  --> fügt nur die Erklärung ein
% \acp{Abk.}  --> gibt Plural aus (angefügtes 's'); das zusätzliche 'p' funktioniert auch bei obigen Befehlen